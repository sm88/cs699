\section{Introduction}
\label{sec1}
\noindent{Software estimation is a crucial element in software engineering and project management. Incorrect software estimation leads to late delivery, surpassing the budget and project failures. According to the International Society of Parametric Analysis (ISPA) and the Standish Group International, the main reasons behind project failures include optimism in conducting software estimation as well as misunderstanding and uncertainty in software requirements. At the inception of each software project, project managers use several techniques to predict software size and effort that will help them learn the cost, required time and the number of staff required to develop a project. Examples of these techniques include Algorithmic Models such as COCOMO\footnote{Constructive cost model}, SLIM\footnote{Software lifecycle management} and SEER-SEM, Expert Judgment, Estimation by Analogy and Machine Learning techniques.\\
In this paper, we present a novel Artificial Neural Network (ANN) model to estimate software effort based on the UCP method. The importance of our model is that it can be used in the early stages of the software life cycle where software
estimation is required and difficult to conduct at this phase. The proposed ANN model takes three inputs which include software size, productivity and project complexity. Software size and productivity are estimated using the UCP model. A new approach to calculate the project complexity of a project is also introduced. To better evaluate the proposed ANN model, we introduce a multiple linear regression model to predict software effort based on three independent variables. We then tested the ANN model against the regression model as well as the UCP model
based on the Mean of Magnitude of error Relative to the Estimate (MMER) and prediction level PRED. Results show that the ANN model outperforms the multiple linear regression model and UCP models based on the MMER criterion by 8\% and 50\% respectively, and thus, can be a competitive model for software effort prediction.
The remainder of this paper is organized as follows: Section \ref{sec2} presents a background of terms used in this paper. Section \ref{sec3} introduces related work whereas Section \ref{sec4} introduces the model's inputs. Section \ref{sec5} illustrates the proposed ANN and multiple linear regression models. In Section \ref{sec6}, the proposed ANN will be evaluated and in Section \ref{sec7}, threats to validity are listed. Finally, Section \ref{sec8} concludes the paper and suggests future work.}
